% XeLaTeX can use any Mac OS X font. See the setromanfont command below.
% Input to XeLaTeX is full Unicode, so Unicode characters can be typed directly into the source.

% The next lines tell TeXShop to typeset with xelatex, and to open and save the source with Unicode encoding.

%!TEX TS-program = xelatex
%!TEX encoding = UTF-8 Unicode

\documentclass[12pt]{article}
\usepackage{geometry}                % See geometry.pdf to learn the layout options. There are lots.
\geometry{a4paper}                   % ... or a4paper or a5paper or ... 
%\geometry{landscape}                % Activate for for rotated page geometry
%\usepackage[parfill]{parskip}    % Activate to begin paragraphs with an empty line rather than an indent
\usepackage{graphicx}
\usepackage{amssymb}

% Will Robertson's fontspec.sty can be used to simplify font choices.
% To experiment, open /Applications/Font Book to examine the fonts provided on Mac OS X,
% and change "Hoefler Text" to any of these choices.

\usepackage{fontspec,xltxtra,xunicode}
\defaultfontfeatures{Mapping=tex-text}
\setromanfont[Mapping=tex-text]{Helvetica}
\setsansfont[Scale=MatchLowercase,Mapping=tex-text]{Gill Sans}
\setmonofont[Scale=MatchLowercase]{Andale Mono}
% 支持中文的排版
\usepackage{xeCJK}
\setCJKmainfont[BoldFont=STSong, ItalicFont=STKaiti]{STSong}
\setCJKsansfont[BoldFont=STSong]{STSong}
\setCJKmonofont{STSong}

\title{Brief Article}
\author{The Author}
\date{}                                           % Activate to display a given date or no date

\begin{document}
\maketitle

\section{Power Transfer Function}
	Microresonators have a Lorentzian power transfer function which is peaked at the resonant wavelength $\lambda _{MR}$. For optical signals carried on wavelength $\lambda_{s}$, the drop-port power transfer can be expressed as (1) and the through-port power transfer can be expressed as (2)[$1$]. When   $k_{e}^{2} + k_{d}^{2} \gg k_{p}^2$ , nearly full power transfer can be achieved at the peak resonance point, and the microresonator will exhibit a low insertion loss. Physical im- plementations show that the insertion loss of a microresonator can be practically lowered to 0.5 dB [$42$].
\begin{equation}
\label{drop-port power transfer} 
\frac{P_{drop}}{P_{in}}=\left ( \frac{2k_{e}k_{d}}{k_{e}^2 + k_{d}^2 + k_{p}^2} \right )^2 \cdot \frac{\delta ^2}{\left ( \lambda _{s} -  \lambda _{MR}\right )^2 + \delta ^2}
\end{equation}

\begin{equation}
\label{through-port power transfer} 
\frac{P_{through}}{P_{in}}=1-  \frac{4k_{e}^2(k_{d}^2+k_{p}^2)}{\left ( k_{e}^2 + k_{d}^2 + k_{p}^2 \right )^2} \cdot \frac{\delta ^2}{\left ( \lambda _{s} -  \lambda _{MR}\right )^2 + \delta ^2}
\end{equation}

\begin{equation}
\label{Parameter Detail} 
k_{e} = k_{d}, k_{e}^2 = 43.36 k_{p}^2
\end{equation}

\section{Micro Resonator Router Power Model}
A optical router consist of Micro resonators. The floorplan of MR router expressed as $A$, $P$, $C$. $A_{ij}$ is the number of active MRs from port $i$ to port $j$. $P_{ij}$ is the number of passive MRs from port $i$ to port $j$.  $C_{ij}$ is the number of crossing loss from port $i$  to port $j$. 

\section{Active/Passive MR Optical Loss Model}
The optical loss in MR mainly consist of active MR loss and passive MR loss. The lambda of VCSEL and the lambda of MR is sensitive to the temperature as .

\begin{equation}
\label{active MR loss} 
L_{active} = 10log(\frac{P_{drop}}{P_{in}})
\end{equation}

\begin{equation}
\label{active MR loss} 
L_{passive} = 10log(\frac{P_{thro}}{P_{in}})
\end{equation}

\section{Router level Optical loss Model}
Router level optical loss is different from the input port and output port. These factors decide the.numbers of active MRs and passive MRs which optical conducting. So the optical loss from port i to port j could be expressed as :
\begin{equation}
\label{active MR loss} 
L_{ij} = A_{ij}\cdot L_{active} + P_{ij}\cdot L_{passive} + C_{ij}*L_{c}
\end{equation}

\section{Network level Optical loss Model}
Network level Optical loss Model is decided by the all optical loss in router and $L_{w}$ optical in wave guide. So the all optical loss in a whole path from source to the destination can be expressed as:
\begin{equation}
\label{active MR loss} 
L(s, d) = \sum_{m=1}^{N}L_{ij}^{m} + L_{WG}
\end{equation}
N is the number of all routers optical conducting from source to destination. $i$ is the input port in  $m$ router and $j$ is the output port in $m$ router. 

% For many users, the previous commands will be enough.
% If you want to directly input Unicode, add an Input Menu or Keyboard to the menu bar 
% using the International Panel in System Preferences.
% Unicode must be typeset using a font containing the appropriate characters.
% Remove the comment signs below for examples.

% \newfontfamily{\A}{Geeza Pro}
% \newfontfamily{\H}[Scale=0.9]{Lucida Grande}
% \newfontfamily{\J}[Scale=0.85]{Osaka}

% Here are some multilingual Unicode fonts: this is Arabic text: {\A السلام عليكم}, this is Hebrew: {\H שלום}, 
% and here's some Japanese: {\J 今日は}.
\end{document}  