% XeLaTeX can use any Mac OS X font. See the setromanfont command below.
% Input to XeLaTeX is full Unicode, so Unicode characters can be typed directly into the source.

% The next lines tell TeXShop to typeset with xelatex, and to open and save the source with Unicode encoding.

%!TEX TS-program = xelatex
%!TEX encoding = UTF-8 Unicode

\documentclass[12pt]{article}
\usepackage{geometry}                % See geometry.pdf to learn the layout options. There are lots.
\geometry{a4paper}                   % ... or a4paper or a5paper or ... 
%\geometry{landscape}                % Activate for for rotated page geometry
%\usepackage[parfill]{parskip}    % Activate to begin paragraphs with an empty line rather than an indent
\usepackage{graphicx}
\usepackage{amssymb}

% Will Robertson's fontspec.sty can be used to simplify font choices.
% To experiment, open /Applications/Font Book to examine the fonts provided on Mac OS X,
% and change "Hoefler Text" to any of these choices.

\usepackage{fontspec,xltxtra,xunicode}
\defaultfontfeatures{Mapping=tex-text}
\setromanfont[Mapping=tex-text]{Helvetica}
\setsansfont[Scale=MatchLowercase,Mapping=tex-text]{Gill Sans}
\setmonofont[Scale=MatchLowercase]{Andale Mono}
% 支持中文的排版
\usepackage{xeCJK}
\setCJKmainfont[BoldFont=STSong, ItalicFont=STKaiti]{STSong}
\setCJKsansfont[BoldFont=STSong]{STSong}
\setCJKmonofont{STSong}

\title{Brief Article}
\author{The Author}
\date{}                                           % Activate to display a given date or no date

\begin{document}
\maketitle

\section{Power Transfer Function}
	Microresonators have a Lorentzian power transfer function which is peaked at the resonant wavelength $\lambda _{MR}$. For optical signals carried on wavelength $\lambda_{s}$, the drop-port power transfer can be expressed as (1) and the through-port power transfer can be expressed as (2)[$1$]. When   $k_{e}^{2} + k_{d}^{2} \gg k_{p}^2$ , nearly full power transfer can be achieved at the peak resonance point, and the microresonator will exhibit a low insertion loss. Physical im- plementations show that the insertion loss of a microresonator can be practically lowered to 0.5 dB [$42$]. 
\begin{equation}
\label{drop-port power transfer} 
\frac{P_{drop}}{P_{in}}=\left ( \frac{2k_{e}k_{d}}{k_{e}^2 + k_{d}^2 + k_{p}^2} \right )^2 \cdot \frac{\delta ^2}{\left ( \lambda _{s} -  \lambda _{MR}\right )^2 + \delta ^2}
\end{equation}

\begin{equation}
\label{through-port power transfer} 
\frac{P_{through}}{P_{in}}=1-  \frac{4k_{e}^2(k_{d}^2+k_{p}^2)}{\left ( k_{e}^2 + k_{d}^2 + k_{p}^2 \right )^2} \cdot \frac{\delta ^2}{\left ( \lambda _{s} -  \lambda _{MR}\right )^2 + \delta ^2}
\end{equation}

\subsection{}


% For many users, the previous commands will be enough.
% If you want to directly input Unicode, add an Input Menu or Keyboard to the menu bar 
% using the International Panel in System Preferences.
% Unicode must be typeset using a font containing the appropriate characters.
% Remove the comment signs below for examples.

% \newfontfamily{\A}{Geeza Pro}
% \newfontfamily{\H}[Scale=0.9]{Lucida Grande}
% \newfontfamily{\J}[Scale=0.85]{Osaka}

% Here are some multilingual Unicode fonts: this is Arabic text: {\A السلام عليكم}, this is Hebrew: {\H שלום}, 
% and here's some Japanese: {\J 今日は}.
\end{document}  